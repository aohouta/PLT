\documentclass[a4paper,12pt]{article}
\usepackage{times}
\usepackage[french]{babel}
\usepackage[utf8x]{inputenc}
\usepackage[T1]{fontenc}
\usepackage{amsmath}
\usepackage{amssymb}
\usepackage{graphicx}
\usepackage{pdfpages}
\usepackage{pdflscape}
\usepackage{listings}
\usepackage{longtable}
\lstset{literate=
{é}{{\'e}}1
{è}{{\`e}}1
{ê}{{\^e}}1
{à}{{\`a}}1
{â}{{\^a}}1
}
\lstset{language=C++,
                basicstyle=\footnotesize,
                keywordstyle=\footnotesize\color{blue},
                otherkeywords={override,nullptr}
}
\definecolor{orange}{rgb}{0.8,0.4,0.0}
\definecolor{darkblue}{rgb}{0.0,0.0,0.6}
\definecolor{cyan}{rgb}{0.0,0.6,0.6}
\lstdefinelanguage{JSON}
{
  basicstyle=\normalsize,
  columns=fullflexible,
  showstringspaces=false,
  commentstyle=\color{gray}\upshape,
  morestring=[b]",
  morestring=[s]{>}{<},
  morecomment=[s]{<?}{?>},
  stringstyle=\color{orange},
  identifierstyle=\color{darkblue},
  keywordstyle=\color{blue},
  morekeywords={string,number,array,object}% list your attributes here
}

\sloppy

\setlength{\topmargin}{0cm}
\setlength{\headsep}{0.in}
\setlength{\headheight}{0.in}
\setlength{\evensidemargin}{0cm}
\setlength{\oddsidemargin}{-1cm}
\textwidth 18cm
\textheight 25cm

\newcommand{\HRule}{\rule{\linewidth}{0.5mm}}


\begin{document}

\thispagestyle{empty}

\begin{titlepage}

\begin{figure}[h]
\begin{center}
\includegraphics[scale=1.2]{Logo_ENSEA_couleur.jpg}
\end{center}
\end{figure}

\vspace*{2.5cm}
\HRule
\vspace{0.5cm}
\begin{center}\textbf{\Huge Projet Logiciel Transversal}\end{center}{\Large \par}

\begin{center}\textbf{\large Patrick Contin, William Duval Bourligueux, Bastien Guillard, Abdel-Oihed Houta}\end{center}{\large \par}

\HRule
\vspace{2cm}

\begin{figure}[h]
\begin{center}
\includegraphics[scale=1]{FFT.png}
\end{center}
\end{figure}


\clearpage

{\small
\tableofcontents
}

\end{titlepage}

\clearpage
\section{Présentation Générale}

\subsection{Archétype}

Le jeu est un tactical RPG, basé sur des jeux tels que Final Fantasy
tactics, Fae tactics ou encore la série des Fire Emblem.

\subsection{Règles du jeu}

Le jeu se déroule sur une map la forme d'une grille, celle-ci voit 
s'affronter 2 équipes de plusieurs personnages. Chaque personnage peut se 
déplacer et interagir (attaquer, soigner, etc\dots) avec les autres sur la map.
La victoire est déclarée quand l'ensemble des membres d'une équipes 
sont KO (PV = 0). \\
Les personnages commencent avec 0 de mana et en gagnent un peu en début de chaque tour, les sorts ont différents couts de mana en fonction de leur puissance.

\subsubsection{Stat des personnages}

\begin{tabular}{|l|l|}
  \hline
  Nom en jeu & Effet \\
  \hline
  \hline
  PV(Point de Vie) & Point de vie du personnage, il est KO s'ils tombent a 0.\\
  \hline
  PM(Point de Mana) & Point de Mana, consommés par les capacités. \\
  \hline
  Attaque & Attaque physique d'un personnage. \\
  \hline
  Armure & Défense physique d'un personnage. Résiste aux dégats physique.\\
  \hline 
  Magie & Puissance d'effets des capacités qui consomme du mana. \\
  \hline 
  Ténacité & Défense magique d'un personnage. Résiste au dégats magique \\
  \hline 
  Vitesse & Permet de déterminer l'ordre des tours \\
  \hline
  Mobilité & Nombre de case pouvant être parcouru en 1 seul tour. \\
  \hline
  Esquive & Chance d'esquive du personnage. \\
  \hline
\end{tabular}

\subsubsection{Classe des personnage}

Les classes sont reparties en plusieurs categories selon leur utilités (degat, tank, support) et leur portées (mêlé et porté).
De base il y aurait 3 classes : 

\begin{itemize}
    \item un guerrier tank : peu de mobilité, vitesse, beaucoup de défense et de vie. Son attaque de base est un coup d'épée au corps a corps, avec un sort de protection, et un sort qui force un ennemi a l'attaquer.
    \item un mage support : peu de mobilité, peu de défense, peu de dégat, moyenne portée, ses sorts permêttent d'augmenter les stats des alliées et de les soignées.
    \item un archer qui fait des dégats : peu de défense, vitesse moyenne, beaucopu de portée et de dégats, un sort qui fait beaucoup de dégat sur une seule cible, et un sort qui fait des dégat de zone.
\end{itemize}

\subsection{Ressources}

Pour les ressources de ce projet, nous avons réaliser la carte du jeu avec le logiciel Tiled 
(Voir dossier res).


\clearpage
\section{Description et conception des états}

\subsection{Description des états}

Un état de jeu a besoin de 3 informations, la carte du actuelle du jeu,
les personnages et leur état, ainsi que les joueurs et leur état.

\subsubsection{L'état de la carte du jeu}

La carte du jeu est composé d'une liste de cellule, qui compose la carte.
Chaque cellule peut être, soit :
\begin{itemize}
  \item Être vide, juste le sol de la case, avec rien dessus
  \item Avoir un personnage dessus, peut importe son état
  \item Avoir un obstacle (arbre, rocher, ou autre)
\end{itemize}

Il y a plusieurs carte de jeu prédéfini.

\subsubsection{L'état du joueur}

Le joueur possède une liste de personnage qu'il possède, avec lesquels il peut jouer.
Il possède aussi un état, qui indique si :
\begin{itemize}
  \item il est encore en jeu
  \item il a gagné sa partie
  \item il a perdu sa partie
  \item il ne joue plus au jeu  (un timer compte, si le joueur prend trop de temps à jouer)
\end{itemize}

\subsubsection{L'état des personnages}

Les personnages de chaque joueur possède un ensemble de statistiques :
\begin{itemize}
  \item Leur point de vie (PV), permet de déterminer combien de coup le personnage peut prendre avant de mourrir
  \item Leur attaque (ATK), permet de déterminer combien de dégats le personnage va faire avec son attaque de base
  \item Leur attaque magique (MAG), permet de déterminer combien de dégats le personnage va faire avec ses sorts
  \item Leur défense magique (RM), permet de réduire les dégats pris par les sorts
  \item Leur défense physique (DEF), permet de réduire les dégats pris par les attaques de base
  \item Leur vitesse (VIT), permet de déterminer l'ordre des personnages
  \item Leur mobilité (MOB), permet de déterminer la quantité de case que le personnage peut se déplacer
  \item Leur esquive (ESQ), permet d'éviter les attaques et les sorts
\end{itemize}

Chaque personnage possède également un nom, un compteurs de tours, ainsi qu'une liste
de sorts qu'il peut lancer. Il possède aussi une liste d'effets qui va être remplie aux cours de la partie,
au fur et a mesure que des effets lui sont appliqué.

\subsubsection{L'état général du jeu}

L'état général du jeu permet de compter le nombre de tours passé, de joueurs encore en jeu.
Ainsi que l'index du joueur qui est en train de jouer, et une liste de tout les personnages encore en jeu.

\subsection{Conception Logiciel}

Le diagramme des classes pour les états est présenté en Figure \ref{uml:state}. En rouge foncé
on distingue la classe "état" principal, alors qu'en rouge plus clair on voit les classes lié aux personnages 
et la carte du jeu. En blance on a les énumerations, qui sont utilisé pour ne pas utiliser de "magic numbers", 
et de permettre une transparence sur l'utilisation des variables et de leur différents états.

\begin{landscape}
\begin{figure}[p]
\includegraphics[width=0.8\paperwidth,angle=270]{StateUML_1.pdf}
\caption{\label{uml:state}Diagramme des classes d'état.} 
\end{figure}
\end{landscape}

\clearpage
\section{Rendu: Stratégie et Conception}

\subsection{Stratégie de rendu d'un état}
La stratégie pour laquelle nous avons choisis d'opter est celle utilisée dans l'exemple de M.Gosselin. Cette stratégie simple est basée sur un développement de bas niveau avec des élément simples comme des textures, couches, etc..\\
En effet, nous avons choisi de représenté un état suivant 3 étapes :\\
\begin{itemize}
\item la map 
\item les personnages
\item les objets\\
\end{itemize}
\par 
\paragraph{Map}
La première étape concerne la map, nous avons utilisé le tutoriel dans la documentation SFML. Pour éviter de recharger la map à chaque coup d'horloge, nous décidons de l'afficher sur une couche (layer). La map sera quadrillé sous forme de matrice sur laquelle on vient attribuer des textures grâce à des tuiles. Ainsi les textures seront fixées et la map ne sera plus rechargée, ce qui permettra d'alléger la charge du CPU et de la carte graphique.\\
\paragraph{Personnages}
Les personnages sont crées grâce à une liste d'objet instanciés avec la classe personnage du state. Ces personnages seront placés sur la map selon leurs positions définies par les attributs de la classe Position du state. Concernant les textures elles seront chargées grâce à des sprites définis dans la bibliothèques SFML. Cela demande plus de ressources que la méthode pour afficher la map mais comme nous avons peu de personnage, ce n'est pas un problème. Le rendu s'actualise en permanence suivant une fréquence que l'on définira par la suite sachant que 50/60 Hz est la fréquence de rafraîchissement habituelle.

\paragraph{Objets}
Les objets seront traités uniquement si la partie map et personnages sont fonctionnelles. Les objets pourront affecter le rendu des personnages (un personnage peu changer de couleur s'il obtient un boost d'attaque) ce qui peut compliqué le code. Ainsi les objets seront traités dernier suivant la même méthode que les personnages (avec des sprites).



\subsection{Conception logiciel}
Le diagramme du rendu est disponible ici : \ref{uml:render}.\\
\par
La classe \textbf{StateLayer} va nous permettre de créer les différentes couches pour les différents niveaux. Elle permet aussi d'avoir un affichage graphiques avec à l'instanciation d'une fenêtre graphique de la bibliothèque SFML. Elle récupère des informations de toutes les autres classes et gère l'ensemble du rendu.\\
\par
La classe \textbf{LoadLayer} va nous permettre de texturer nos couches avec l'utilisation de quads, ensemble de 4 coins formant un rectangle, qui permet de cadrier la map pour y associer des textures.\\
\par
La classe \textbf{Tiles} permet de gérer les tuiles. Mais surtout, elle permet de récupérer les ressources sous forme d'images que nous associons aux tiles. Sans ceci la texture est impossible.\\


\begin{landscape}
\begin{figure}[p]
\includegraphics[width=0.8\paperwidth,angle=270]{render.pdf}
\caption{\label{uml:render}Diagramme des classes de rendu.} 
\end{figure}
\end{landscape}

\clearpage
\section{Règles de changement d'états et moteur de jeu}

\subsection{Règles}

\subsubsection{Actualisation}

L'engine est complètement indépendant du render, le moteur de jeu
ne sert qu'à mettre a jour l'état du jeu, en fonction des commandes exterieurs 
(Clavier, souris ou serveur), et des règles automatiques, qui sont vérifer a chaque 
changement d'état. Le moteur de jeu effectue une mise a jour de l'état a chaque
commande exterieur, les règles automatiques n'ont besoin de s'activer qu'uniquement 
dans le cas d'un changement d'état (jeu tour par tour). Il existe aussi une règle 
autonome qui doit s'incrémenter entre les tours, afin de déterminer l'ordre de passage
des tours des différents personnages. Cette commande sera appeler par une horloge.

\subsubsection{Règles extérieurs}

Les règles extérieurs sont provoqué par les cliques de souris a diffrénts endroit, 
ou les ordres reçue par le réseau.

\begin{itemize}
  \item Selectionner un personnage, afin d'afficher les statistiques du personnage
  \item Déplacer les personnages
  \item Attaquer un ennemi
  \item Lancer un sort (pas encore implémanter)
\end{itemize}

\subsubsection{Règles automatiques}

Les règles automatiques sont les checks qui sont lancés a chaque fois que l'ont veut faire
une action, afin de savoir si l'action est valide, elles sont éxécuter en fonction de 
la commande extérieur qui a été exécuté auparavant. Un autre type de règle automatique est 
la barre d'action, qui détermine l'ordre des tours, cette barre d'actions est une commande
qui va s'éxécuter a intervalles réguliers.

Les différentes règles suivant les actions sont, pour :

\begin{itemize}
  \item Incrémenter la barre d'action :
  \begin{itemize}
    \item Tester si un personnage est arrivé aux maximum de la barre
    \item Si un personnage est arrivé au maximum : passer en mode tour par tour,
    lui permettre de jouer et lui donner son mana.
  \end{itemize}
  \item Selectionner un personnage :
  \begin{itemize}
    \item Chercher les informations sur le personnage selectionner
    \item Afficher les informations sur le personnage selectionner
  \end{itemize}
  \item Déplacer un personnage :
  \begin{itemize}
    \item Vérifier la disponibilité des cases 
    \item Calculer le pathfinding
    \item Effectuer le déplacement
  \end{itemize}
  \item Attaquer un personnage
  \begin{itemize}
    \item Tester la possibilité de l'attaque (porté, ligne de vue)
    \item Calculer les positions relatives, pour les dégats directionnels
    \item Calculer les dégats infligés (défense, crit et autre)
    \item Appliquer les dégats au personnage
    \item Tester si le personnage est mort
  \end{itemize}
  \item Lancer un sort
  \begin{itemize}
    \item Obtenir les informations sur le type de sort
    \item Obtenir les informations sur la cible 
    \item Réaliser l'action en fonction du sort (si c'est une attaque, on attaque, un soin, on soigne\dots)
  \end{itemize}
\end{itemize}

\clearpage
\subsection{Conception logiciel}

Le diagramme de classe de moteur de jeu est reprsenté sur la figure %\ref{uml:engine}.
Le jeu repose sur un patron de conception de type Command.

La classe commande est une classe abstraite qui parentes toutes les autres classe commande.

Les classes commandes, ce sont les classes qui représentent les différentes commandes :

\begin{itemize}
  \item ShowInfoCommand, prend une case en argument et renvoie les informations sur le personnage qui s'y situe 
  \item AttackCommand, prend une case en argument et lance une attaque sur cette case, en vérifiant les autres règles
  \item MoveCommand, prend une case en argument est déplace, si possible, le personnage sur la case
  \item SpellCommand, prend une case en argument, est lance le sort sur cette case 
  \item TickCommand, incrémente les ticks de tour, afin de fire avancer la barre des tours
\end{itemize}


\begin{landscape}
\begin{figure}[p]
\includegraphics[width=0.8\paperwidth,angle=270]{engine.pdf}
\caption{\label{uml:engine}Diagramme des classes de moteur de jeu.} 
\end{figure}
\end{landscape}


\section{Intelligence Artificielle}

\subsection{Stratégies}
\subsubsection{Intelligence aléatoire}
L'intelligence artificielle que nous implémentons consiste à sélectionner un personnage aléatoirement dans la liste des personnages encore vivant. Après avoir sélectionner le personnage, l'intelligence va choisir aléatoirement une action entre "déplacer le personnage" et "attaquer un personnage".\\
\subsubsection{Intelligence heuristiques}
Pour avoir un meilleur comportement que le hasard, nous décidons d'implémenter un ensemble d'heuristiques afin que l'intelligence artificielle puisse répondre au problème suivant :\\
\begin{center}
\textbf{Tuer tous les personnages de l'adversaire}.
\end{center}
Pour commencer nous avons choisis les heuristiques suivantes :\\
\begin{itemize}
\item Si un personnage est à portée, on l'attaque !
\item Sinon, on cible un personnage qui possède moins de 30\% de ses PVMax pour s'en approcher.
\item Sinon, on cible un personnage qui possède une défense inférieure à notre attaque.
\end{itemize}
\subsubsection{Intelligence avancé}
Pour avoir une intelligence qui réfléchit par-elle même, nous avons opté pour une intelligence utilisant
un arbre de recherche. 
\\ Notre IA avancé va commencer par faire une liste de tous les personnages faisant partie de l'équipe ennemie,
puis, dans une copie du "state", elle va essayer de se rapprocher de chacun de ces ennemis et de les attaquer, 
on va ensuite laisser l'IA heuristique jouée quelques tours, et on note le résultat, si un allié est mort, on a une
mauvaise note, si un ennemi est mort, alors on a une bonne note, on prend alors le coup qui a permis d'avoir
la meilleure note, et on le joue.
\clearpage

\clearpage
\subsection{Conception logiciel}
\subsubsection{IA}
\paragraph{RandomIA}
Pout générer de l'aléatoire nous utilisons la fonction \textbf{rand()} de la librairie \textit{<cstdlib>}. Elle renvoie une série de chiffres générer par des algorithmes donc on aura pas un nouveau chiffre après chaque compilation.
\paragraph{HeuristiqueIA}
Nous avons introduit un système de \textbf{node}, les nodes sont crées en parcourant la map et chaque cellule sera associée à une node, elles sont distinguées selon si un personnage est présent dans la case.\\
De plus à chaque node et associée un alentours qui est une liste de node "libre" autour d'elle. Ceci permet aux personnages de se déplacer tout en esquivant les autres personnages. 
\paragraph{DeepIA}
Pour l'IA avancén nous nous sommes inspiré de l'IA heuristique, en y ajoutant un système d'état temporaire
de simulation, et un système de note, qui permet de classifier les différents coups.
\begin{landscape}
\begin{figure}[p]
\includegraphics[width=0.6\paperheight,angle=270]{ai.pdf}
\caption{\label{uml:ai}Diagramme des classes d'intelligence artificielle.} 
\end{figure}
\end{landscape}


\section{Modularisation}
\label{sec:module}

\subsection{Organisation des modules}

\clearpage
\subsection{Conception logiciel}


%
%\begin{landscape}
%\begin{figure}[p]
%\includegraphics[width=0.9\paperheight]{module.pdf}
%\caption{\label{uml:module}Diagramme des classes pour la modularisation.} 
%\end{figure}
%\end{landscape}

\end{document}

\documentclass[a4paper,12pt]{article}
\usepackage{times}
\usepackage[french]{babel}
\usepackage[utf8x]{inputenc}
\usepackage[T1]{fontenc}
\usepackage{amsmath}
\usepackage{amssymb}
\usepackage{graphicx}
\usepackage{pdfpages}
\usepackage{pdflscape}
\usepackage{listings}
\usepackage{longtable}
\lstset{literate=
{é}{{\'e}}1
{è}{{\`e}}1
{ê}{{\^e}}1
{à}{{\`a}}1
{â}{{\^a}}1
}
\lstset{language=C++,
                basicstyle=\footnotesize,
                keywordstyle=\footnotesize\color{blue},
                otherkeywords={override,nullptr}
}
\definecolor{orange}{rgb}{0.8,0.4,0.0}
\definecolor{darkblue}{rgb}{0.0,0.0,0.6}
\definecolor{cyan}{rgb}{0.0,0.6,0.6}
\lstdefinelanguage{JSON}
{
  basicstyle=\normalsize,
  columns=fullflexible,
  showstringspaces=false,
  commentstyle=\color{gray}\upshape,
  morestring=[b]",
  morestring=[s]{>}{<},
  morecomment=[s]{<?}{?>},
  stringstyle=\color{orange},
  identifierstyle=\color{darkblue},
  keywordstyle=\color{blue},
  morekeywords={string,number,array,object}% list your attributes here
}

\sloppy

\setlength{\topmargin}{0cm}
\setlength{\headsep}{0.in}
\setlength{\headheight}{0.in}
\setlength{\evensidemargin}{0cm}
\setlength{\oddsidemargin}{-1cm}
\textwidth 18cm
\textheight 25cm

\begin{document}

\thispagestyle{empty}

\begin{titlepage}

\vspace*{2cm}

\begin{center}\textbf{\Huge Projet Logiciel Transversal}\end{center}{\Large \par}

\begin{center}\textbf{\large Patrick Contin, William Duval Bourligueux, Bastien Guillard, Abdel-Oihed Houta}\end{center}{\large \par}

\vspace{2cm}

%\begin{figure}[h]
%\begin{center}
%\includegraphics[width=\textwidth]{exemple.png}
%\caption{\label{pacmangame}Exemple du jeu}
%\end{center}
%\end{figure}

\clearpage

{\small
\tableofcontents
}

\end{titlepage}

\clearpage
\section{Présentation Générale}

\subsection{Archétype}

Le jeu est un tactical RPG, basé sur des jeux tels que Final Fantasy
tactics, Fae tactics ou encore la série des Fire Emblem.

\subsection{Règles du jeu}

Le jeu se déroule sur une map la forme d'une grille, celle-ci voit 
s'affronter 2 équipes de plusieurs personnages. Chaque personnage peut se 
déplacer et interagir (attaquer, soigner, etc\dots) avec les autres sur la map.
La victoire est déclarée quand l'ensemble des membres d'une équipes 
sont KO (PV = 0). \\
Les personnages commencent avec 0 de mana et en gagnent un peu en début de chaque tour, les sorts ont différents couts de mana en fonction de leur puissance.

\subsubsection{Stat des personnages}

\begin{tabular}{|l|l|}
  \hline
  Nom en jeu & Effet \\
  \hline
  \hline
  PV(Point de Vie) & Point de vie du personnage, il est KO s'ils tombent a 0.\\
  \hline
  PM(Point de Mana) & Point de Mana, consommés par les capacités. \\
  \hline
  Attaque & Attaque physique d'un personnage. \\
  \hline
  Armure & Défense physique d'un personnage. Résiste aux dégats physique.\\
  \hline 
  Magie & Puissance d'effets des capacités qui consomme du mana. \\
  \hline 
  Ténacité & Défense magique d'un personnage. Résiste au dégats magique \\
  \hline 
  Vitesse & Permet de déterminer l'ordre des tours \\
  \hline
  Mobilité & Nombre de case pouvant être parcouru en 1 seul tour. \\
  \hline
  Esquive & Chance d'esquive du personnage. \\
  \hline
\end{tabular}

\subsubsection{Classe des personnage}

Les classes sont reparties en plusieurs categories selon leur utilités (degat, tank, support) et leur portées (mêlé et porté).
De base il y aurait 3 classes : 

\begin{itemize}
    \item un guerrier tank : peu de mobilité, vitesse, beaucoup de défense et de vie. Son attaque de base est un coup d'épée au corps a corps, avec un sort de protection, et un sort qui force un ennemi a l'attaquer.
    \item un mage support : peu de mobilité, peu de défense, peu de dégat, moyenne portée, ses sorts permêttent d'augmenter les stats des alliées et de les soignées.
    \item un archer qui fait des dégats : peu de défense, vitesse moyenne, beaucopu de portée et de dégats, un sort qui fait beaucoup de dégat sur une seule cible, et un sort qui fait des dégat de zone.
\end{itemize}

\subsection{Ressources}

Pour les ressources de ce projet, nous avons réaliser la carte du jeu avec le logiciel Tiled 
(Voir dossier res).


\clearpage
\section{Description et conception des états}

\subsection{Description des états}


\subsection{Conception Logiciel}


%\begin{landscape}
%\begin{figure}[p]
%\includegraphics[width=0.9\paperheight]{state.pdf}
%\caption{\label{uml:state}Diagramme des classes d'état.} 
%\end{figure}
%\end{landscape}

\clearpage
\section{Rendu: Stratégie et Conception}

\subsection{Stratégie de rendu d'un état}


\subsection{Conception logiciel}

%\begin{landscape}
%\begin{figure}[p]
%\includegraphics[width=0.9\paperheight]{render.pdf}
%\caption{\label{uml:render}Diagramme des classes de rendu.} 
%\end{figure}
%\end{landscape}

\clearpage
\section{Règles de changement d'états et moteur de jeu}

\subsection{Règles}

\clearpage
\subsection{Conception logiciel}


%\begin{landscape}
%\begin{figure}[p]
%\includegraphics[width=0.9\paperheight]{engine.pdf}
%\caption{\label{uml:engine}Diagramme des classes de moteur de jeu.} 
%\end{figure}
%\end{landscape}


\section{Intelligence Artificielle}

\subsection{Stratégies}

\clearpage
\subsection{Conception logiciel}


%\begin{landscape}
%\begin{figure}[p]
%\includegraphics[width=0.9\paperheight]{ai.pdf}
%\caption{\label{uml:ai}Diagramme des classes d'intelligence artificielle.} 
%\end{figure}
%\end{landscape}


\section{Modularisation}
\label{sec:module}

\subsection{Organisation des modules}

\clearpage
\subsection{Conception logiciel}


%
%\begin{landscape}
%\begin{figure}[p]
%\includegraphics[width=0.9\paperheight]{module.pdf}
%\caption{\label{uml:module}Diagramme des classes pour la modularisation.} 
%\end{figure}
%\end{landscape}

\end{document}
